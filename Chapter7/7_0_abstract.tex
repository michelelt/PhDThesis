%\section{Abstract}
%%Self-contained abstract of no more than 100 words MA HO VISTO CHE NEGLI ALTRI PAPER NE USANO CIRCA 180, outlining the aims, scope and conclusion of the paper. Three to five keywords must be included.
%%Ora siamo a 209
%
%In this work I consider the design of a Free Floating Car Sharing (FFCS) system based on Electric Vehicles. I face the problems of finding the optimal placement of charging stations, and the design of smart car return policies, i.e, how many and where to place charging stations, and whether to ask or not customers to return the car to a charging pole.
%
%I leverage actual data containing rentals performed by Car2Go customers described in chapter \ref{chap:2_dataset}. I obtain information for several months worth of actual trips in the city of Turin, this use case.
%Via trace driven simulations, I replay the exact same trips while simulating electric car based FFCS, to accurately gauge battery discharging and recharging.
%With this, I compare different charging station placements, also driven by optimisation algorithms. Moreover, I observe the impact of collaborative or selfish car return policies.
%
%Results are surprisingly: just as few as 13 charging stations (52 poles) guarantee a fleet of 377 vehicles running in a 1 million inhabitant city to work flawlessly, with limited customer's discomfort.
%I believe this data driven methodology may help researchers and car sharing providers discerning different design solutions in order to find a lower-bound reference to setup an charging infrastructure for a FFCS system.

This chapter is mostly taken from \textit{"Data driven optimization of charging station placement for EV free floating car sharing"} published in the $21^{st}$ International Conference on Intelligent Transportation Systems in November 2018 (\cite{maui}) and \textit{"Free floating electric car sharing design: Data driven optimisation"} published on Pervasive and Mobile Computing journal in April 2019 (\cite{coccaopt})
  