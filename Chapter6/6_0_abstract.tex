%\section{Abstract}
%
%Car sharing is a popular means of transport in smart cities, with the free floating paradigm which lets the customers autonomously pick and drop available cars. This freedom impose several challenges and trade-off in the design of an electrical version of this type of car sharing. 
%
%In this paper I study the design of an electric free floating car sharing system for four cities by leveraging actual rental traces. I analyze via accurate simulations through the simulator described in chapter \ref{chap:5_simulator} the impact of i) the charging station placement, and ii) return policies.
%The traces composed by millions of rentals in different cities world wide located, captures the non-stationary and highly dynamic nature of usage patterns of customers.
%
%Considering charging station placement, it is possible to demonstrate how it is better to place charging stations within popular parking area (e.g., downtown), even if parking duration is short. I show how by smart placing and handful number of charging stations (below 10\% of city zones) the system can self-sustain itself. 
%These results are obtained also thanks to customers' willingness to collaborate by voluntarily returning the car to a nearby charging station.
%
%The results help car sharing provider comparing possible alternative design solution, i.e.,  the adoption of simple relocation policies that would move cars that need a charge only, a promising solution to limit discomfort for customers due to rerouting. The same could be achieved by considering giving incentives to customers. 

This chapter refers mostly two works: \textit{"Free floating electric car sharing in smart cities: Data driven system dimensioning"} published in the IEEE International Conference on Smart Computing (SMARTCOMP) in July 2018 (\cite{taormina}) and \textit{"Free Floating Electric Car Sharing: A Data Driven Approach for System Design"} published in the IEEE Transactions on Intelligent Transportation Systems journal in August 2019 (\cite{coccacar})
