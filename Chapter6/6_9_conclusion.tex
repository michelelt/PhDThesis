\section{Conclusion}
\label{sec:conclusion}

Designing an electric vehicle free floating car sharing system leads to many interesting problems and trade-offs.
In this work, we built on actual rental traces to study via accurate simulations the impact of i) the charging station placement, and ii) return policies. 
Considering charging station placement, I demonstrated that it is better to place charging stations within popular parking area (e.g., downtown), even if parking duration is short.

We have shown that a FFCS solution with electric vehicles can almost sustain itself, even with very few charging stations (8-10\% of city zones). These results are obtained also thanks to customers' collaboration by returning the car to a nearby charging station, or whenever the battery level drops below a target threshold.

Car sharing providers shall take into account the trade-off between usability, costs and benefits for the customers. 
The results hint for possible alternative design solution, i.e.,  the adoption of simple relocation policies that would move cars that need a charge only, a promising solution to limit discomfort for customers due to re-routing enforcement. This could be achieved by considering giving incentives to customers.

