In this thesis, I proposed a data-driven pipeline to study the conversion from internal combustion engine electric vehicles in FFCS. Strongly relying on real FFCS data, optimization algorithms, and machine learning techniques, I depicted a pipeline able to evaluate the performances of electric FFCS in hot and cold start-up scenarios. 

In the first part of my thesis, I described the data gathering, scraping real users' ride from operative combustion engine FFCS. Then I characterized the users' habits in 2017 for two FFCS providers: car2go and Enjoy. I highlighted the difference between one-way CS, two-ways CS, and FFCS, which pointed out how the users prefer flexibility over costs. Once all the data are consolidated I developed an electric FFCS mobility simulator. It is an event-based trace-driven simulator. It takes as input the list of rentals events, the infrastructure setups (in terms of charging station placements, car consumptions), and the provider's fleet management policies. It replicates users' patterns and returns as output the metrics describe system performance (complete battery depletion, cars unavailability) and users' discomfort caused car plugging procedures. Armed with this, I initially investigated the best electric FFCS setup varying case of study city (and thus users' patterns), finding how an electric FFCS is sustainable with few charging stations. Next, I improved the solution of the previous step by running several optimization algorithms maximizing both system resilience and minimizing the users' discomfort. Then, I moved my attention to users' ride predictability computing how weather and socio-economics aspects are related to the users' trips. Finally, I simulated possible electric FFCS growth targeting which are the most sensible parameters and how profitable an electric FFCS may be.

The results are surprising. In particular, the data-driven charging station placement showed how the current demand might be sustained with only 104 charging poles in a city of about 1 million inhabitants like Turin. This estimate may be still reduced with ad-hoc algorithms at 72 poles. Finally, the scalability studies (made open source) may lead FFCS providers and policy makers aware of all the benefits of electric mobility.

However, the research presents several limitations. In particular, the dataset is composed of passive measurements, the reason for that it is impossible to distinguish users' trips and unallocated missed trips. The software does not really know when the users did not find an available vehicle nearby. Moreover, the real travelled distances are unknown, due to the anonymization related to privacy. This should not really impact macro analyses but it becomes sensible when it necessary to perform neighbourhood-grained optimizations. This dissertation does not take into account fleet management (relocation). This topic opens several research threads, that need a pragmatic formulation and sophisticated upgrade to the software. Finally, the recent COVID-19 pandemic changed drastically both private and shared users' patterns opening new questions on mobility sustainment in smart cities.

In general, I believe my thesis proposes a pragmatic methodology to study the electric (r)evolution of shared mobility. It will be very challenging and stimulating for me to face the unsolved problem in my future research career.

