\section{Introduction}
\label{sec:2_1_intro}

Mobility is one of the challenges to solve in our society and in cities, where eco-sustainability is becoming more and more important. Regulators and policy makers are positively looking into ``smart'' approaches to improve the current status of their urban network.  The ability to collect data, is the first step to take informed decisions. Unfortunately, getting information about mobility patterns and human driving habits is not easy because of both technical challenges and privacy issues.
%
To this extent, in this chapter I describe the possibility of harvesting data openly exposed on the Web to obtain information about mobility habits in cities, and make it available to the players by using a smart-platform. In particular on car sharing platforms and mapping and direction services.

Car sharing refers to a model of car rental where customers rent a car for a short period of time, usually for a few hours or less. One of its most interesting systems is the so called \textit{Free-Floating Car Sharing (FFCS)} system. The peculiarity of this system is that customers can pick and drop the car wherever in a geo-fence area. The most famous company is car2go which is present in 25 cities and 8 different countries, both in Europe and North America. Moreover, other country-based providers exists like \textit{Enjoy} in Italy, operative in 6 cities.

This chapter describes how \tool collects, processes, augments, and stores data in a data lake, make the data available for further analyses. In particular, I build two crawlers to collect data from the \textit{car2go} and \textit{Enjoy} platforms\footnote{\url{www.car2go.com}, \url{enjoy.eni.com}}. Every minute, the crawler checks which cars are currently available. Every time a given car ``disappears'', it records the booking start time. The same booking ends when the crawler sees the car available back on the system. Some booking are actual ``rental'' in case the car moved from the prior parking position to another. Ingenuity must be used, e.g., to filter GPS fix issues (which may erroneously let a car ``move''), or to handle possible data collection issues (e.g., the website going down, or some cars undergoing in maintenance), or platform design (e.g., synchronous or asynchronous updates).

%In total, \tool collected data for \mc{365} days, from December 10th 2016 to January 31st 2017 We observed more than 104,000 \textit{bookings} and 86,000 \textit{rentals} for car2go, and 93,000 \textit{bookings} and 81,000 \textit{rentals} for Enjoy. 
%\mc{rivedere dati su quanti bookings ci sono}

In total, \tool collected about XXX of trips in X cities for car2go, working from 2017 to 2018. Instead, considering Enjoy, I get about YYYY rides from YY cities from 2017 to 2020
\mc{rivedere dati su quanti bookings ci sono}. The source code of \tool for research purposes.\footnote{\url{github.com/MobilityPolito/}}


The reminder of this chapter is structured as follows: section~\ref{sec:2_3_data_acquisition} describes in details the raw data structure and the data flow from providers' API to the middle stage. Section~\ref{sec:2_4_data_normalization} describes how the software implements the data raw elaboration to trips records and their storage into data lake . \mc{continuare con altre sezione se ne aggiungo}





%Although, free floating car sharing is one of the most debated topic in research about mobility, the shared and open data suitable for research are quite rare. 
%
%Mainly, the free floating car sharing providers do not will share their traffic data in order to not expose to competitors import insights on their operative condition. Secondly policy makers imposes to companies the users' privacy, forbidding vehicle tracking while users travels.




