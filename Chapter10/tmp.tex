


@article{SPREI2019127,
title = "Free-floating car-sharing electrification and mode displacement: Travel time and usage patterns from 12 cities in Europe and the United States",
journal = "Transportation Research Part D: Transport and Environment",
volume = "71",
pages = "127 - 140",
year = "2019",
note = "The roles of users in low-carbon transport innovations: Electrified, automated, and shared mobility",
issn = "1361-9209",
doi = "https://doi.org/10.1016/j.trd.2018.12.018",
url = "http://www.sciencedirect.com/science/article/pii/S1361920918303158",
author = "Frances Sprei and Shiva Habibi and Cristofer Englund and Stefan Pettersson and Alex Voronov and Johan Wedlin",
keywords = "Shared mobility, Free-floating car-sharing, Electric vehicles, Usage patterns, Travel time, Alternative trips",
abstract = "Free-floating car-sharing (FFCS) allows users to book a vehicle through their phone, use it and return it anywhere within a designated area in the city. FFCS has the potential to contribute to a transition to low-carbon mobility if the vehicles are electric, and if the usage does not displace active travel or public transport use. The aim of this paper is to study what travel time and usage patterns of the vehicles among the early adopters of the service reveal about these two issues. We base our analysis on a dataset containing rentals from 2014 to 2017, for 12 cities in Europe and the United States. For seven of these cities, we have collected travel times for equivalent trips with walking, biking, public transport and private car. FFCS services are mainly used for shorter trips with a median rental time of 27 min and actual driving time closer to 15 min. When comparing FFCS with other transport modes, we find that rental times are generally shorter than the equivalent walking time but longer than cycling. For public transport, the picture is mixed: for some trips there is no major time gain from taking FFCS, for others it could be up to 30 min. For electric FFCS vehicles rental time is shorter and the number of rentals per car and day are slightly fewer compared to conventional vehicles. Still, evidence from cities with an only electric fleet show that these services can be electrified and reach high levels of utilization."
}









@book{Tan:2018,
 author = {Tan, Pang-Ning and Steinbach, Michael and Karpatne, Anuj and Kumar, Vipin},
 title = {Introduction to Data Mining (2Nd Edition)},
 year = {2018},
 isbn = {0133128903, 9780133128901},
 edition = {2nd},
 publisher = {Pearson},
}

@Inbook{Jin2010,
author="Jin, Xin
and Han, Jiawei",
editor="Sammut, Claude
and Webb, Geoffrey I.",
title="K-Means Clustering",
bookTitle="Encyclopedia of Machine Learning",
year="2010",
publisher="Springer US",
address="Boston, MA",
pages="563--564",
isbn="978-0-387-30164-8",
doi="10.1007/978-0-387-30164-8_425",
url="https://doi.org/10.1007/978-0-387-30164-8_425"
}



















@electronic{26_c2g_history,
  howpublished={\url{https://www.car2go.com/EU/career/}},
  note = {Accessed: 2020-07-09},
}












@electronic{Car2goNews,
  author = {Ronan Glon},
  title = {{Car2Go car-sharing service shutting down in the U.S. after reality check}},
  howpublished = {\url{https://www.digitaltrends.com/cars/car2go-drive-now-share-now-shutting-down-in-america-in-2020/}},
  note = {Accessed: 2019-12-27},
  year= {2019}
}

@electronic{ShareNow,
  title = {{Share Now, the Daimler and BMW-owned car sharing service, is exiting North America and three European cities}},
  howpublished = {\url{https://au.news.yahoo.com/share-now-daimler-bmw-owned-212223266.html?soc_src=social-sh&soc_trk=ma}},
  note = {Accessed: 2019-12-27}
}



@electronic{financialChargingStations,
author = {UCLA Luskin Center for Innovation and UCLA Anderson School of Management},
title = {Financial Viability Of Non-Residential Electric Vehicle Charging Stations},
  note = {Accessed: 2020-07-09},
year = {2012}
}







@electronic{down,
  author = {SHARENOW},
  title = {Service Ending February 29th},
  howpublished = {\url{https://www.share-now.com/us/en/important-update/}},
  note = {Accessed: 2020-01-31},
  year = {2019}
}