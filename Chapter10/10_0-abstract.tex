%\section{abstract}
%\label{sec:10_0_abstract}
%
%In this chapter I analyze which are the design options that would impact a free floating electric car sharing system performance and costs, studying how the system would scale with an increase in the intensity of the demand.
%I consider the case study of the city of Turin, for which I leverage hundred of thousands of actual rentals from a (combustion-based) car sharing system to derive an accurate demand model. Armed with this, I consider the transition to  electric cars and the need to deploy a charging station infrastructure.
%
%Using a realistic simulator, I present the impact of system design options, like the number of charging poles, their allotment, and the number of cars. I first consider performance indicators, like fraction of satisfied demand and working hours system has to spend to bring to charge vehicles. Then I map these figures into revenues and costs, projecting economical indicators. 
%At last, I investigate the scalability of the whole system, i.e., how performance and costs scale when the demand increases.
%The results show that concentrating the charging stations in key places is instrumental to optimize car distribution in the city to better intercept the demand.
%Considering system scalability, the charging infrastructure must intuitively grow proportionally with the mobility demand.
%Interestingly instead, the fleet size can grow much slower, showing some nice economy of scale gains.

This chapter is mostly taken from \textit{"On Scalability of Electric Car Sharing in Smart Cities"} (\cite{barulli2020scalability}) published in the 2020 IEEE International Smart Cities Conference (ISC2) in September 2020.
