% !TEX encoding = UTF-8 Unicode
% !TEX TS-program = lualatex
% !BIB TS-program = biber

\documentclass[%
   corpo=12pt, % optional; default:= 10pt
   oneside, % recommended
   tipotesi=scudo,
   mybibliostyle, % only if biblio is typeset with a personal style
  numerazioneromana, % not necessary in a real thesis
   ]{toptesi}
   

\begin{filecontents*}{\jobname.xmpdata}
\Author{Michele Cocca}
\Title{Conversion of FFCS ICE fleet to EV: a data-driven approach}
\Subject{Doctoral dissertations in the SCUDO doctoral school}
\Keywords{PDF\sep
          PDF/A\sep
          ISO 19005\sep
          LaTeX\sep
          PhD Thesis\sep
          Engineering\sep
          SCUDO}
\Publisher{Politecnico di Torino}
\end{filecontents*}
%%%%%%% -----------------------------------------------------------
   
%%%% Use the following package if and only if you want to produce
%%%% an archivable document according to standard PDF/A-1b
%%%% No need to load package imakeidx, because it has already been
%%%% loaded by the specific module toptesi-scudo.
\usepackage[a-1b]{pdfx}
\usepackage[linesnumbered,ruled,norelsize]{algorithm2e}



\newtheorem{definition}{Definition}
%%%%%% Read the English documentation of TOPtesi in order to check
%%%%%% the special attention needed to produce ISO compliant
%%%%%% archivable files
%%%%

\ifPDFTeX
    \usepackage[utf8]{inputenc}% 
    \usepackage[T1]{fontenc}
\fi

\errorcontextlines=9% more information on the console in case of errors

%%%% Specify fonts here; chose one among these fonts by leaving just
%%%% one line without initial comment character.
%%%% With LuaLaTeX or XeLaTeX don't change fonts
\ifPDFTeX % using pdflatex
    \usepackage{lmodern} % Default
    %\usepackage{newtxtext,newtxmath}% Times eXtended for text and math
    %\usepackage{fourier}% Utopia, Helvetica and "monospace = ?"

\else % using lualatex (or xelatex)
% Here we use the Libertinus serif, sans serif, monospaced and math fonts;
% they are alla available with a complete up-to-date TeXLive installation.
% Without specifying any OpenType font, the Latin Modern OpenType ones are
% used by default; try commenting the setting for Libertinus Mono, run with
% LuaLaTeX and see what happens; you might prefer to keep the comment sign
% in this line.
    \usepackage{fontspec}
    \defaultfontfeatures{Ligatures=TeX}
    \setmainfont{Libertinus Serif}
    \setsansfont{Libertinus Sans}
    \setmonofont{Libertinus Mono}[Scale=MatchLowercase]
    \usepackage{unicode-math}% add special math stile option here
                             % for example [style=ISO]
% define one math font 
    \setmathfont{Libertinus Math}%
\fi


\usepackage{kantlipsum,mwe} % to produce dummy text and dummy figures
\usepackage{booktabs}



\makeindex[intoc]% collect material to index

%%%%%%%%%%%%%%%%%%%%%%%%%%%%%%%%%%%%%%%%%%%%%%%%%%%%%%%%%%%%%%
% This conditional code is an example of using a different bibliography
% style from the one predefined by the class.
% The usage of the conditional code (also with a different contents) is
% discouraged, because the predefined numbered bibliography style is the
% one generally used in scientific publications.
% Even if discouraged, it is not forbidden.
\ifmybibstyle 
  \usepackage[autostyle]{csquotes} % necessary for biblatex
  \usepackage[backend=biber,
              style=philosophy-classic,
              scauthors=all,
              sorting=nyt,
              natbib]{biblatex} % LaTeX specific bibliography handler
\fi 
\addbibresource{\jobname.bib}% bibliographic data base(s)
% It is recommended to name a single or the primary .bib file with the same
% name as the thesis master file: the macro \jobname contains that name.
% It is possible lo use a comma separated list of bibliographical databases
% but extreme care must be paid to the fact that all .bib files have
% actually different names, and that the citations key are really distinct
% among alla databases, not simply within a single database.
%%%%%%%%%%%%%%%%%%%%%%%%%%%%%%%%%%%%%%%%%%%%%%%%%%%%%%%%%%%%%%
% The following is to be loaded as the end of the preamble if one wants
% to use hyperlinks and/or urls to be typed within the thesis,
% except that after loading hyperref very few commands may be
% issued. One is the \includeonly command with its list of files;
% other packages may be loaded after hyperref, only if their
% documentation says so; some of these critical packages, but they are
% not the only ones, involve Right to Left languages or other
% oriental languages.
%
% Distinguish the hyperref call from the hyperref setup so as
% to avoid option clashes with other packages that may invoke 
% hyperref with different options.

\unless\ifcsname ver@hyperref.sty\endcsname\usepackage{hyperref}\fi
\hypersetup{%
    pdfpagemode={UseOutlines},
    bookmarksopen,
    pdfstartview={FitH},
    colorlinks,
    linkcolor={blue},
    citecolor={blue},
    urlcolor={blue}
  }
%%%%%%%%%%%%%%%%%%%%%%%%%%%%%%%%%%%%%%%%%%%%%%%%%%%%%%%%%%%%%%

%%%% The \includeonly argument should preferably be written with
%%%% one file name per line, so that by commenting or uncommenting
%%%% some lines a selective compilation may be executed.
\includeonly{%
Chapter1/1_introduction,%
Chapter2/2_datacollection,%
Chapter3/3_datacharact,%
Chapter4/4_cs_comparison,%
Chapter5/5_simulator,
Chapter6/6_4cities,
Chapter7/7_cs_optimization,
Chapter8/8_prediction,
%Chapter9/9_relocation,
Chapter10/10_scalability,
Chapter11/11_conclusion
%Appendix1/appendix1,
%Appendix2/appendix2,%
}
%%%%%%%%%%%%%%%%%%%%%%%%%%%%%%%%%%%%%%%%%%%%%%%%%%%%%%%%%%%%%%
\newcommand{\tool}{\textit{UMAP}\xspace}
\newcommand{\MGM}[1]{#1}
\newcommand{\removelatexerror}{\let\@latex@error\@gobble}
\newcommand{\reviewed}[1]{{\color{black}{#1}}}
\newcommand{\mc}[1]{{\color{purple}{[mike: #1]}}}
\usepackage[thinlines]{easytable}

\DeclareMathOperator*{\argmax}{arg\,max}
\DeclareMathOperator*{\argmin}{arg\,min}


\ifPDFTeX \usepackage{indentfirst}\fi
\raggedbottom

\begin{document}

% The contents of this ThesisTitlePage environment may be written
% in a configuration file named exactly the same as the thesis main
% file, but with extension .cfg. If similar commands with different
% data are written within this environment, such data prevail on
% those read from the configuration file.
\begin{ThesisTitlePage}
% Use the optional command to set a different School logo
% Its is possible to used this command several times; each time
% a new different  Institution logo  may be added. In general
% just the ScuDo logo is sufficient; for dissertations made in
% cooperation with the University of Turin, its logo may be added
% with a second instance of the \PhGschoolLog statement. With
% dissertations supported by the INRiM, this institution logo may
% be added. Such logos (in PDF format) may be obtained from the
% ScuDo web server.
\PhDschoolLogo{scudo}% Fake logo for this example
%\PhDschoolLogo{Logo-ScuDo-blu} % just the ScuDo logo
%\PhDschoolLogo{Logo-ScuDo-blu,Logo-INRIM} %for dissertations made in cooperation with INRIM
%\PhDschoolLogo{Logo-ScuDo-blu,Logo-INFN} %for dissertations made in cooperation with INFN
%\PhDschoolLogo{Logo-ScuDo-blu,Logo-UniTo} %for dissertations made in cooperation with the University of Turin
% Doctorate course name; mandatory
\ProgramName{Electronic and Telecommunication Engineering}
% Cycle ordinal number; optional.
% You can write 29.th, or 29\ap{th}, or 29\textsuperscript{th}, or ...
\CycleNumber{33.th}
% PhD candidate name; mandatory
\author{Michele Cocca}
% Dissertation title; mandatory
\title{Electric Revolution and\\Free Floating Car Sharing}
% Dissertation subtitle: optional.
% It might be useful only if the actual full title is too long.
\subtitle{A Data Driven Methodology for System Design}
% The supervisor(s) label; optional; default value "Supervisor:".
% You can change it to plural as in this example, where the colon has
% been eliminated.
%\NSupervisor{Supervisor}{Supervisors}
%
% The SupervisorNumber may contain a value such as 0, 1, or any
% number higher than 1. If 0 is specified, no label is typeset
% over the supervisor(s) list; if 1 is specified then the singular
% form is used: if any value higher than 1 is specified, the plural
% form is used.
\SupervisorNumber{1}
% List of supervisors with academic title, name(s), surname(s),
% and function; mandatory
\SupervisorList{%
    Prof.~Marco Mellia, Supervisor\\
    }
% Name of the examining committee: optional. 
% Default value "Doctoral Examination Committee"
%\Nexaminationcommittee{Doctoral examination committee}
% List of the  examining committee members: mandatory if the above label
% is not empty.
\ExaminerList{%
Prof.~A.B., Referee, University of \dots\\
Prof.~C.D., Referee, University of \dots\\
Prof.~E.F., University of \dots\\
Prof.~G.H., University of \dots\\
Prof.~I.J., University of \dots}
% Name of the institution where the examination takes place; optional.
% Default value: "Politecnico di Torino"
%\Nlocation{Politecnico di Torino}
% Examination date: mandatory, although the way to write it is free.
\ExaminationDate{\today}
% Disclaimer with signature; optional. Default text as
% in the following lines. 
\Disclaimer{%
\noindent I hereby declare that, the contents and organisation of this dissertation constitute my own original work and does not compromise in any way the rights of third parties, including those relating to the security of personal data.	
}
%\Signature{%
%\begin{flushright}
%\parbox{0.5\textwidth}{\centering
%\dotfill\\
%Mario Rossi\\
%Turin, February 29, 2123
%}
%\end{flushright}}
\end{ThesisTitlePage}
%%%%%%%%%%%%%%%% Everything else necessary in the thesis title
%%%%%%%%%%%%%%%% page and in the copyright page is supplied by
%%%%%%%%%%%%%%%% the default values.
%%%%%%%%%%%%%%%% If you enter an explicit disclaimer, you can
%%%%%%%%%%%%%%%% typeset other material before the disclaimer
%%%%%%%%%%%%%%%% formula; use the necessary spacing on order
%%%%%%%%%%%%%%%% to separate the formula from other text. 
%%%%%%%%%%%%%%%% 
%%%%%%%%%%%%%%%% For example you might want to write the formal
%%%%%%%%%%%%%%%% statement for a particular licence, provided the 
%%%%%%%%%%%%%%%% licence allows Open access; not necessarily all uses
%%%%%%%%%%%%%%%% of the thesis should be allowed, but the minimum
%%%%%%%%%%%%%%%% is the reading access.

%%%%%%%%%%%%%%%% The next two lines are metadata for a normal PDF file.
%%%%%%%%%%%%%%%% For ISO archivable PDF/A-1b metadata, they must
%%%%%%%%%%%%%%%% be entered only in the form used in the file
%%%%%%%%%%%%%%%% named in the filecontents* environment as shown
%%%%%%%%%%%%%%%% at the beginning of this file.

%\subject{How to typeset a doctoral thesis suitable for defence in almost any country and university.}
%\keywords{{pdfLaTeX} {LuaLaTeX} {XeLaTeX} {PhD doctoral programs} {PhD dissertation} {Politecnico di Torino}} 

\summary%\sommario
%This is where you write your abstract \dots\ (Maximum 4000 characters, i.e. maximum two pages in normal sized font, typeset with the thesis layout).
%The abstract environment is also available, but  \texttt{\string\summary} is preferred because it generates an un-numbered chapter. The abstract environment is more suitable for articles and two column typesetting without a separate title page.
\mc{intro}

In my thesis, I presented a methodology to address, in different cases of studies, all the challenges related to the conversion of combustion engine cars to electric vehicles in FFCS. In particular, the main drivers focus on a profitable and technically sustainable system setup able to guarantee a flexible and appealing mobility service to an increasing customer audience.

In the first part of my thesis, I described the software I developed in order to scrape from the web real combustion engine FFCS, from two providers: car2go and Enjoy. The data collection car2go data collection lasted from \mc{december 2016 to january 2018, collecting more than 28 million users' bookings spread in 25 cities} while the Enjoy data collection phase started \mc{may 2018 and lasted until BOH}, collectin about \mc{BOH} bookings in 6 cities. 

Then, I characterized both datasets in Turin, one of the cities in which both FFCS providers work. Initially, I defined a filtering criterion to remove outliers from the dataset and identify only real users' ride: entries reporting a trip where the user realistically moved. These analyses allowed me to identify geo-spatial user's patterns. After that, I compared the car2go customer's pattern with the one-way and two-way car-sharing system. The results showed how users prefer more flexible services like FFCS or one-way car sharing.

Once the data are consolidated, I developed (i) a methodology to assign a charging station to a city by looking at users' pattern; (ii) system policies to manage the fleet when the vehicle SoC may not guarantee a trip; (iii) an event-based trace-driven simulator able to replicate the recorded trips in an electrified scenario evaluating the feasibility of each configuration. 

At this point I studied in Berlin, Milan, Turin, and Vancouver (cities having different traffic patterns) different electric FFCS setup. The best logic to place a charging station is to cover the zones recording more parkings: increasing the charging station presence in those zones, the probability to find a free plug increases too. Then, I showed how an electric FFCS is sustainable with few charging stations (8-10~\% of city zones) if the users accept to park the car in a charging station when their ride finishes nearby or when the SoC goes below a security threshold. Finally, I investigate different charging station setups. The best results come by balancing the spread of charging stations: concentrated all plugs in a unique hub or spread them all above the city leads to poor performances in terms of system sustainability and user's discomforts.

To still reduce the number of charging stations to have a sustainable electric FFCS, I compared several optimization algorithms. My results showed how a Genetic Algorithm is able to find a better solution able to shrink the minimum amount of resources to sustain the same mobility demand.

After that, I moved my attention to the users' rentals predictability. The main goal is to understand how different open-data sources could impact on the recorded FFCS users' rental. I used as a case of study the city of Vancouver, augmenting the rentals dataset with the weather and socio-economic open data. Initially, I compared several time-series forecasts to predict the users' demand in the short and medium-term. The Random Forrest regression produced better accuracy and results interpretability. Then I correlated the socio-economics features characterizing each Vancouver neighborhood to FFCS demand and again, the Random Forrest regression clearly showed a subset of socio-economic features more correlated to the FFCS users' rentals.

Finally, I questioned the system scalability figuring out several scenarios having increasing demand. I used a model able to synthesize users' demand by looking only at the geo-spatial users' rentals. By varying the electric FFCS setup and simulating the new scenario I pointed out how a linear increase in the demand intensity leads to a fleet sub-linear increase. Finally, I projected those considerations in euros proofing how electric FFCS has rooms for economic growth. 

\begin{dedication}

\textit{I would like to dedicate this thesis to my loving parents} 

{\normalsize 
The dedication very seldom is a proper thing to do; in some countries it is very common, while in other countries it is done for imitation of other people habits. 

The sentence used above clearly is an example of something very common, but it is  useless. Of course we all love our beloved parents, but it is not necessary to ``engrave it in stone''.
\par}
\end{dedication}
%\end{dedica}

%%%%%%%%% If you want these two lists, uncomment the following line
%\tablespagetrue\figurespagetrue % 

%%%%%%%%% Table of contents and optionally the tables and figures lists
\allcontents

\mainmatter %all the above is front matter; here begins the main matter

% !TEX root = ../toptesi-scudo-example.tex
% !TEX encoding = UTF-8 Unicode
%***********************************************************************
%*********************************** First Chapter 
%***********************************************************************

\chapter{Introduction}  %Title of the First Chapter
\label{chap:1_introduction}
    \graphicspath{{Chapter1/}}

%% Increasing of population in big cities and relative mobility demand
Nowadays, the 55\% world population is strongly concentrated in urban centers and the authors of \cite{UNfuture} forecasted, in 2018, an increase until 68\% before 2050. In this scenario, one of the problem  afflicts big conurbations is mobility. In the past decades, the economic growth were sustain by the flourishing automotive industry. \cite{8_matas2008changes} demonstrated how the presence of private cars in an European Countries like Spain will follow the demographic and correlated mobility demand increase. 

%% cost related to traffic congestion (emission related to TC  and pollutant mortality costs)
The increase of private vehicles derived from the increased mobility demand leads to an higher probability of traffic congestion, especially in rush hours. A lot of past and recent studies demonstrated how traffic jams have a negative effect on the public health. Indeed  \cite{12_leiriao2020environmental} showed how the thin dust emission are strongly related with traffic, measuring air pollutant before and during a truck strike. Moreover,  \cite{13_levy2010evaluation} computed the monetized value of PM2.5-related emission attributable to congestion. The results are shocking: the public health may cost up to \$17 billion in 2030 in the United States. Another aspect that decreases the quality of life in congested city is the noise pollution. In particular, \cite{14_mehdi2011spatio} and \cite{15_jacyna2017noise} stated how the noise can reach very dangerous level and causes permanent damages to people living close to trafficked areas. Finally, the land use  related to cars presence is another critical point that policy maker must manage.

In the past years, the several different regulatory boards addressed the congestion increasing and optimizing the road network. According to several works like \cite{10_sweet2011does, 16_hymel2010induced, 17_naess2012traffic} adding more resources, attracts more car, actually feeding the traffic congestion burden.

%% shared economy and CS
The raise of shared economies, made possible groundbreaking revolution in mobility too. In particular, this paradigm allows to customer to  access to  goods and services thorough a peer-to-peer instance.  It includes the shared creation, production, distribution, trade and consumption of goods and services by different people and organisations. 

Thanks to this particular business model, the so called car sharing born. The very first implantation included shared car purchase among people who did not were able to buy the car. During the economic growth after the WWII, the shared cars increasingly appeal and several business started across all the Europe, like \cite{4_shaheen1999short} and \cite{5_shaheen1998carsharing} tells.

With the Internet advent, car sharing still improved the service's users' experience. In particular the providers' infrastructure made possible reserve and release the car just using web-based application run on smartphones. 
In contraposition with classical car rental model with per-day fares, the new technologies made possible bill the users' with new fares based mainly on the time spent driving. 

Finally, it is possible provide a definition of \textbf{car sharing}: \textit{a car rental model, where the users pay only for the time spent driving, all the other costs, like petrol, insurance and maintenance are in charge to the provider. The reservation and return producers are possible without physical presence in provider front-end office}.

During the year, several typologies of car sharing born. A definition of all car sharing typologies is needed in order to distinguish all the implementation. The initial concept to introduce is the operative area: the 

\begin{itemize}
	\item \textbf{Two-way car sharing}: The users can reserve a car in one of the several ad-hoc spot spread around the operative area, but they are forced to return it in the same spot. It allows the providers to limit the fleet operations to meet the demand but heavily limits the service flexibility reducing the freedom degree of the users.
	
	\item \textbf{One-way car sharing}: The user can reserve a car in one of the company-owned parking station, but contrary to the previous one, he/she returns the car in a different parking station. This approach increases the flexibility of the service but it requires more load balance by the provider.
	
	\item \textbf{Free floating  sharing (FFCS)}: The users can pick and release the car \textit{everywhere} within an operative area. It means that shared cars can be parked in common parking like private cars. In this case the provider does not build an infrastructure but has a more fleet management.
\end{itemize}


Since first implementation of firsts model of car sharing intended as business model, the debate on the environmental sustainability of this service was very spirited. In particular first studies at the beginning of the millennium, like \cite{1_fellows2000economic} conducted a business study of car sharing setup in UK. It pointed out how the environmental benefits due to the decrease of private cars circulation like $CO_2$ emission and land use were achieved, with a fraction of the cost needed for a new road scheme.

Later, \cite{2_huwer2004public} addressed the doubt about the possible public transport abandon by customers due to the presence of this new alternative in urban mobility. Counter-intuitively the author proofed that the car sharing can fill the users' mobility demand between public transport and to drive a private car providing the benefits of driver a private car without all the fixed costs that characterize it. Thus, including car sharing in the ecosystem of urban mobility offers, the users will be more prone to avoid travel with their own car and finally incrementing the audience of public transport users.

The groundbreaking event in shared motility was the launch of car2go, in 2008 in Germany. This was one of the the first completely automatized Free Floating Car Sharing providers. As previously mentioned Free Floating Car Sharing allows users' to pick and release the car with any kind of constraint. This flexibility attracted more customers: car2go, started in few cities in Germany, extended its services in 2017 in more than 26 cities in Europe and North America. Recent study, like \cite{9_jochem2020does}, still confirms that the presence of free floating car sharing with other free floating shared vehicle like bike or scooter leads to a decrease of car ownership and thus an increase of quality of life in urban centres.

However, it is possible to think a step forward in environmental sustainability. Indeed, the majority of the cars belonging to car2go have a internal combustion engine. The electric revolution we are experimenting nowadays may lead several improvement to this service too. The benefits due to mobility electrification are well known. In particular, some works like \cite{3_oxley2012pollution} and \cite{4_mao2012achieving} confirms that electric vehicles can still reduce the concentration of pollutant in big cities.

This consideration is at the basis of this work. Indeed, the main research question is:
\begin{quote}
	\centering
	\textbf{It is possible to design an electric car sharing system keeping the free floating paradigm, namely without forcing the users to park the car at the end of each ride?}
\end{quote}

In shared mobility scenario, the the burden of the recharging operations play a fundamental role in terms of service attractiveness. For example, a Telsa Model S can require between 13 and 17 hours for a complete recharge if plugged to a domestic power supply. \footnote{\url{https://www.tesla.com/it_IT/support/home-charging-installation}}, which is an unacceptable time if compared to the few minutes that a petrol fill up requires. 

It follows that, the amount and the electric charging station placement play a key role into this kind of system scenario. If the system is correctly sized, it will possible to have a more customer-centric service able to provide to the customers  a \textit{flexible} shared mobility experience.

In order to tackle this problem, in this thesis I designed a methodology able replicate the shared mobility demand in an electrified scenario, defining the Key Performances Indicators (KPIs) able to state the sustainability of an electric Free Floating Car Sharing System. The complete pipeline will provide useful insight on the charging station placement, the perceived users discomfort related to the plugging operation and customers management policy and the whole system profitability.


I structured my thesis as follow. In chapter \ref{chap:2_dataset} I described the software architecture allowed to collect about 35 million users' rides from two Free Floating Car Sharing Provider. Then, in chapter \ref{chap:3_charact} I characterized the two services and the users' habits using a Turin as case of study city. After that, in chapter \ref{chap:4_cs_comparison} I compared different implementation of car sharing, in order to provide different users' patterns. In chapter \ref{chap:5_simulator} I describe the core of the whole thesis. It is an ad-hoc trace-driven electric free floating simulator. It takes in input the lists of users' ride and replicate the same demand in a electrified scenario. Here I define as well all the KPIs that measures the system efficiency. Chapter \ref{chap:6_4cities} discusses the charging station placement algorithm driven by the collected data comparing the performances of those placements with the simulation outputs. Chapter \ref{chap:7_cs_optimization} compares different optimizations algorithms the charging station placement pointed out by the previous chapter. Then, chapter \ref{chap:8_prediction} starts to study the predicability of the demand, discussing which temporal and socio-economic features may influence the Free Floating Car Sharing Demand. Subsequent ally \ref{chap:10_scalability} compares the system performances tuning some key inputs like the demand intensity, the fleet size and the infrastructure projecting the results on the profit plan from a business point of view. Finally \ref{chap:11_conclusion} concludes the thesis.















\input{Chapter1/1_1_research_question}


% !TEX root = ../toptesi-scudo-example.tex
% !TEX encoding = UTF-8 Unicode
%***********************************************************************
%***********************************Second Chapter
%***********************************************************************

\chapter{Data Acquisition Pipeline}
\label{chap:2_dataset}
	\graphicspath{{Chapter2/}}

%\section{Abstract}
%Free Floating Car Sharing (FFCS) is a popular business model based on the shared-economy paradigm. The users can pick and drop the cars within a given operative area paying only for the time spent driving. This system implies front-end-crew-free reservation and releasing procedure, possible through a FFCS provider's web application. It exposes the coordinates of each available car and making them bookable by only one tap.
%
%In this chapter, I describe how continuously monitoring two FFCS providers and how I created a dataset of FFCS trips. To do that,  I developed the \textit{Urban Mobility Analysis Platform} (\tool) able to fetches data from car sharing platforms in real time. Secondly, \tool processes the data to extract advanced information about driving patterns and user's habits. To extract information, \tool augments the data available from the car sharing platforms with mapping and direction information fetched from other web platforms. This information is stored in a data lake where historical series are built, and later analyzed using analytics modules easy to design and customize. 
%
%In total \tool collected \mc{trips and bookinfs per c2g and engjoy for each city}

This chapter refers mostly my paper " \textit{UMAP: Urban mobility analysis platform to harvest car sharing data} \cite{ciociolaumap}, presented in at 2017 IEEE SmartWord conference




\section{Introduction}
\label{sec:2_1_intro}

Mobility is one of the challenges to solve in our society and in cities, where eco-sustainability is becoming more and more important. Regulators and policy makers are positively looking into ``smart'' approaches to improve the current status of their urban network.  The ability to collect data, is the first step to take informed decisions. Unfortunately, getting information about mobility patterns and human driving habits is not easy because of both technical challenges and privacy issues.
%
To this extent, in this chapter I describe the possibility of harvesting data openly exposed on the Web to obtain information about mobility habits in cities, and make it available to the players by using a smart-platform. In particular on car sharing platforms and mapping and direction services.

Car sharing refers to a model of car rental where customers rent a car for a short period of time, usually for a few hours or less. One of its most interesting systems is the so called \textit{Free-Floating Car Sharing (FFCS)} system. The peculiarity of this system is that customers can pick and drop the car wherever in a geo-fence area. The most famous company is car2go which is present in 23 cities and 8 different countries, both in Europe and North America. Moreover, other country-based providers exists like \textit{Enjoy} in Italy, operative in 6 cities.

This chapter describes how \tool collects, processes, augments, and stores data in a data lake, make the data available for further analyses. In particular, I build two crawlers to collect data from the \textit{car2go} and \textit{Enjoy} platforms\footnote{\url{www.car2go.com}, \url{enjoy.eni.com}}. Every minute, the crawler checks which cars are currently available. Every time a given car ``disappears'', it records the booking start time. The same booking ends when the crawler sees the car available back on the system. Some booking are actual ``rental'' in case the car moved from the prior parking position to another. Ingenuity must be used, e.g., to filter GPS fix issues (which may erroneously let a car ``move''), or to handle possible data collection issues (e.g., the website going down, or some cars undergoing in maintenance), or platform design (e.g., synchronous or asynchronous updates).

%In total, \tool collected data for \mc{365} days, from December 10th 2016 to January 31st 2017 We observed more than 104,000 \textit{bookings} and 86,000 \textit{rentals} for car2go, and 93,000 \textit{bookings} and 81,000 \textit{rentals} for Enjoy. 
%\mc{rivedere dati su quanti bookings ci sono}

In total, \tool collected about 27 million of trips in 23 cities for car2go, working from 2017 to 2018. Instead, considering Enjoy, I get about 6\,6 million rides from 6 cities from May 2017 to June 2019. The source code of \tool for research purposes.\footnote{\url{github.com/MobilityPolito/}}


The reminder of this chapter is structured as follows: section~\ref{sec:2_3_data_acquisition} describes in details the raw data structure and the data flow from providers' API to the middle stage. Section~\ref{sec:2_4_data_normalization} describes how the software implements the data raw elaboration to trips records and their storage into data lake. Section~\ref{sec:2_5_data_analysyes} introduces the final stage of data collection pipeline: a model able to perform several statistics on the collected data. Finally, section~\ref{sec:2_6_conclusions} concludes the chapter.





%Although, free floating car sharing is one of the most debated topic in research about mobility, the shared and open data suitable for research are quite rare. 
%
%Mainly, the free floating car sharing providers do not will share their traffic data in order to not expose to competitors import insights on their operative condition. Secondly policy makers imposes to companies the users' privacy, forbidding vehicle tracking while users travels.






\section{Data Acquisition}
\label{sec:2_3_data_acquisition}
\begin{figure}[h!]
\centering
%left top right top
 \includegraphics[trim=3cm 4.5cm 17cm 7.2cm,clip, width=0.95\columnwidth]{figures/framework_schema.pdf}
 \caption{\tool overview\label{fig:2_3_c2_framework}}
\end{figure}


In this section, I provide a description of \tool structure. Figure~\ref{fig:2_3_c2_framework} depicts the architecture of \tool, composed by a first module for the data acquisition, by a second module for data normalization and integration, and then a third  module for the data analysis.

The first module consists in the data acquisition from the car sharing platforms of interest. These typically expose information about cars' location when available for rental through a web-service approach. 

For this module I design two crawlers, one for the car2go and one for the Enjoy car sharing platforms. They retrieve, at each time instant, which cars are available in a given city.

While car2go offers public APIs~\cite{car2goAPI}, Enjoy does not provide to users such a service. For this reason I reverse engineer the Enjoy web portal. By leveraging the Chrome Developer Tools, I investigate the information exchanged with the Enjoy web portal while asking the list of available cars. Through this analysis, the software obtains both the URL used to request the list of available cars, and how fetch the data for a specific city.
Both system return the currently available cars using a JSON file.

Each time the system downloads a JSON, a \textit{snapshot} describing which cars are parked and ready for rental. Basically, the \textit{snapshot} is a list containing all cars and their attributes.

In a nutshell, a car is described by the car sharing web-service as an object annotated by several information, like plate, vehicle identification number (VIN), location, fuel level, model, etc. 
All the data represented in this object is useful for the customers e.g., to choose which car to rent.
This object is only present if the car is available, i.e., it is parked and free for a rental. Its state changes over time. In particular, a car disappears when a customer reserves and rents it, and then it reappears when the customer ends the rental (likely in a different location).


At each time $t$, the software gets the JSON snapshot $S$ listing the available cars. 
%We take a snapshot \textit{$S$} every minute to balance aggressiveness of the crawler, and time resolution. 
%At each time t, we obtain from each platform the JSON snapshot S detailing the available cars. 
The sampling period has been set to one minute, to balance aggressiveness of the crawler and a reasonable time resolution.
$S$ describes each available car with several fields, some of them being in common between the considered companies, but in general with different format.
For this study, I collect each car unique identifier and current geo-location indication.
These are obtained from the \textit{VIN} or \textit{plate} field, and the \textit{coordinates} field which describes the \textit{longitude} and the \textit{latitude} of the in-car GPS used to localize it when  parked.\footnote{The GPS coordinates are only available if a car is parked and available. There is no risk for users' privacy during rentals. In addition no user's identifier is exposed. Therefore data is totally anonymized as there is no means to know who booked a car.}
In addition to these fields, the car sharing JSON description may provide other information, e.g., the \textit{street address} corresponding to the coordinates, the \textit{fuel} level, the \textit{car interior status} the \textit{engine type}, etc. Since each platform uses its own data and format, I design a data integration step to have common names for fields containing the same information, if present.

\section{Data Normalization and Integration}
\label{sec:2_4_data_normalization}
In this second module I illustrates hoe \tool processes and consolidate each snapshot to obtain \emph{parking} and \emph{bookings} periods for each car. A \emph{parking} is time where the car is available for a user ride. On the other hand a \emph{bookings} is time elapsing two parking where the car is not tracked by the system. The intuition is to track the availability of each car on the car sharing platform, and rebuild the historic parking and booking periods over time: when a customer books a car, the latter ``disappears'' from the system. The framework records this event, with the initial time and position of a new booking. When the customer ends the booking, the car ``reappears'' in the system. The software records this event, with the final time and position of the booking. For the same car, a new parking period starts.

Harvested data is unstructured, and may grow large. Thus I leverage on \textit{MongoDB}, a NoSQL document-based database. A MongoDB database includes a set collections, i.e., groups of documents. Each document is a set of key-value pairs, organized in a JSON structure. The schema-less structure of MongoDB fits well in this work, because it can handle in the same collection documents defined with different key-value pairs. I decide to rely on such a system as I can easily manage the different field structures of providers, car2go and Enjoy in this use case. In addition, MongoDB offers a great integration with Python through the \textit{pymongo} module.

Four different collections compose the MongoDB data lake:  \textit{ActiveBookings}, \textit{ActiveParkings}, \textit{PermanentBookings}, and \textit{PermanentParkings}. 
\textit{ActiveBookings} and \textit{ActiveParkings} are collections used to store information about the current status of cars (currently booked or parked respectively). These are temporary structures that make it easier to query each car last observed status, and update it. These are also instrumentals for a real-time analysis of the system, e.g., to count how many cars are currently booked or available.
\textit{PermanentBookings} and \textit{PermanentParkings} collections store the history of past state of cars, for past bookings and parkings, respectively.

For the documents in the bookings collections I augment information by inserting also the expected route driving time, and the public transportation duration on the same origin-destination pair. These two piece of information are obtained through the Google Directions API using the initial and the final coordinates as indication of the path.

The most important fields in the \textit{ActiveBookings}, and the \textit{PermanentBookings} collections are:
\begin{itemize}
\setlength\itemsep{0.1em}
\item \textit{CarID}: the unique identifier of the car;
\item \textit{InitTime}: the initial time of the booking;
\item \textit{FinalTime}:  the final time of the booking;
\item \textit{InitCoords}:  the GPS coordinates of the booking star location, i.e., where the users picked up the car;
\item \textit{FinalCoords}:  the GPS coordinates of the parking location where the car was dropped at the end of the booking;
\item \textit{DrivingTime}: The duration of the trip, expressed in seconds, as estimated by Google Directions API, following the best path;
\item \textit{PublicTransportTime}: The duration is expressed as arrival time of the best public transport trip, as estimated by Google Directions API, minus the \textit{InitTime};
\end{itemize}

Instead, the \textit{ActiveParkings} and the \textit{PermanentParkings} collections are characterized by the following fields:
\begin{itemize}
\setlength\itemsep{0.1em}
\item \textit{CarID}: the unique identifier of the car
\item \textit{InitTime}: the initial time of the parking
\item \textit{FinalTime}:  the final time of the parking
\item \textit{Coordinates}: the GPS coordinates of the parking spot 
\end{itemize}

\input{Chapter2/pseudo}


I implemented an algorithm to extract booking and parking periods from snapshots, whose workflow is described in the pseudocode in Figure.~\ref{fig:3_2_c2_pseudocodeCarInfoUpdate}. Here I describe each  step.

I consider as inputs the snapshot $S$ and the current timestamp $t$.
Then I create a copy in the list $AP$ of parked cars observed in the previous snapshot (as stored in the $ActiveParkings$ collection) -- line 1. I need the $AP$ list to detect the cars that disappeared, i.e., have been booked at time $t$. This will be back explained later.

For each car $car_j$ in the current snapshot $S$, I check if the car is present in the $AP$ list. 
If so, it means that it did not change its status, i.e., it is still parked. Therefore, the car is removed from the $AP$ list, and nothing is changed -- lines 3-4.
Otherwise, either the car has been parked in this snapshot and the previous booking has finished, or the car is a new car added to the fleet. In both cases a new parking starts and I create a new document in the $ActiveParkings$ collection -- line 7. The \textit{new Parkings} function creates a new document, sets the $InitTime$ and $Coordinates$ keys as current timestamp and car GPS coordinates.
%with the current timestamp and the car GPS coordinates.

I next check if $car_j$ is present in the $ActiveBookings$ collection. If so, the car was booked until the previous snapshot and now it is back available. I thus finalize the previous booking and update its statistics. In particular, the tool sets the $FinalCoords$ and $FinalTime$ fields using the current car $coordinates$ and timestamp -- line 9-10. Next, I check if this booking includes an actual rental by checking if the initial position and final position differ -- line 11-12. Recall indeed that customers may simply book a car but not finalize the rental. Specifically, Enjoy (car2go) offers a grace period of 15 (20) minutes during which no charge is applied for a booking.

In case of an actual rental, I fetch the best path by i) car and ii) public transport from the  $InitPosition$ to the $FinalPosition$ of the rental. I leverage the Google Directions API for this -- line 13-14.\footnote{\url{https://enterprise.google.com/intl/it/maps/products/mapsapi.html}}
It is important to take into account that, while querying the public transportation time, the Google Directions API returns two pieces of information: how long the public transport takes to go from the initial to the final position, and the estimated arrival time. It is fundamental to use this second information because  it includes the time the user spends to reach the bus stop and wait for the bus. This is crucial, e.g., at night, when the first public transport solution may be available only several hours later.

After having processed all cars in the current snapshot, I iterate over the remaining cars in the $AP$ list. Those are the ones that were present in the previous snapshot, but not in the current, i.e., the ones the new bookings. Finally, the software adds to the previous parking period by setting the $FinalTime$ in the $ActiveParking$ collection -- line 21-22. At last, the tool creates a new booking via the \textit{new Booking} function -- line 23.


\begin{table}
	\setlength{\tabcolsep}{2.3pt}
	\centering
	\caption{Overview of car2go's data}
	\begin{tabular}{lcccc}
		\hline
		City &  City Size [$km^2$]\footnote{} & Population \footnotemark[\value{footnote}] &  Avg. Fleat & Bookings\\ 
		\hline
		\hline
		Columbus			  & 17 & 892k & 187 & 186k \\
		Florence 	  			& 34 & 372k & 220 & 333k \\
		Denver 		  			 & 36 & 727k & 312 & 348k \\
		Austin 		   			  & 31 & 964k & 315 & 377k \\
		Frankfurt 			   & 40 & 701k & 242 & 505k \\
		Toronto 				 & 49 & 3120k & 400 & 536k \\
		Amsterdam 			& 38 & 854k & 314 & 573k \\
		Montreal 				& 49 & 1704k & 429 & 606k \\
		New York City 	   & 77 & 8522k & 500 & 739k \\
		Turin 					    & 47 & 874k & 396 & 868k \\
		Munich 					 & 61 & 1464k & 478 & 916k \\
		Washington DC 	  & 64 & 705k & 563 & 919k \\
		Stuttgart 				 & 59 & 632k & 486 & 1001k \\
		Seattle 				   & 92 & 744k & 710 & 1134k \\
		Calgary 				  & 47 & 1239k & 552 & 1176k \\
		Rome 					  & 65 & 2837k & 582 & 1240k \\
		Rheinland 			   & 82 & 1688k & 648 & 1421k \\
		Vienna 					  & 72 & 1915k & 688 & 1702k \\
		Madrid 					 & 42 & 3233k & 424 & 2092k \\
		Milan 					   & 77 & 1396k & 776 & 2223k \\
		Hamburg 			  & 82 & 1833k & 812 & 2561k \\
		Vancouver 			 & 78 & 631k & 977 & 2701k \\
		Berlin 					  & 125 & 3769k & 1009 & 3091k \\
		%\\ \hline
		\hline
		\label{tab:2_3_datasummary}
	\end{tabular}
\end{table}
\footnotetext{\url{wikipedia.org}}
I let \tool scrape car2go's data  \mc{prendere le date}. In total it is possible to count about 27 million bookings spread in 23 cities. The table \ref{tab:2_3_datasummary} reports a brief resume of all the bookings, parkings present in the data lake. 

\mc{devo prendere le date anche di enjoy}



\section{Data Analysis}
The third and final stage is the data analysis phase in which analytics modules query the MongoDB and obtain statistics. I rely on the Python programming language with Pandas and the GeoPandas libraries to deal with the data, the city zone definitions, provided by transport engineers as a shapefile, a popular geospatial vector data format, and the Geographical Information Systems (GIS) for the spatial analyses. I choose Python as it offer a large number of  libraries that easily interact with the different technologies like GIS, maps and MongoDB. In particular the usage of GeoPandas allow me to easily perform geographic analysis and split the city in many areas (or zones) of any possible shapes. I present more detailed characterization in chapter \ref{chap:3_charact}.


\section{Conclusions}
In this chapter I described the software pipeline, named \tool I used to harvest and store data from real FFCS providers. 

The first stage explains the data structures and how I snapshots the system getting all the car ready for a ride. 

The seconds step illustrates the algorithm that compares consecutive snapshots detecting car status variation. Here I introduced the two fundamental car status I use to describe each car history: \textit{parkings} and \textit{bookings}. 

The third briefly opens several scenarios on the analyses of this data.



\include{Chapter3/3_datacharact}
% !TEX root = ../toptesi-scudo-example.tex
% !TEX encoding = UTF-8 Unicode
%***********************************************************************
%***********************************Third Chapter
%***********************************************************************

\chapter{Characterizing Client Usage Patterns and Service Demand for Car-Sharing Systems}
\label{chap:4_cs_comparison}
	\graphicspath{{Chapter4/}}


%\section{Abstract}
%\label{sec:4_0_abstract}

%The understanding of the mobility on urban spaces is useful for the creation of smarter and sustainable cities. However, getting data about urban mobility is challenging, since only a few companies have access to accurate and updated data, that is also privacy-sensitive. 
%
%In this work, we characterize three distinct car-sharing systems which operate in Vancouver (Canada) and nearby regions, gathering data for more than one year. The study uncovers patterns of users' habits and demands for these services. 
%We highlight the common characteristics and the main differences among car-sharing systems. Finally, we believe our study and data is useful for generating realistic synthetic workloads.

This chapter refers mostly the paper \textit{Characterizing client usage patterns and service demand for car-sharing systems} \cite{VancouverCS}, published on the Journal Information Systems, available online since October 11, 2019. My contribution is mainly focused in all the aspects about car2go analyses. 

%Danilo's declaration
%The work I present in this chapter has been developed with the DataBase an Data Mining Group (DBDMG) at the Politecnico di Torino. I joined this group as a network expert to helped them on the validation of the SeLINA tool. In particular, I offered my skill to verify if SeLINA is able to highlight network traffic anomalies. My contributions will be presented in Sec. 5.7 related to the study of the YouTube anomaly previously highlighted in Sec. 2.6.1, and in the analysis of peer-to-peer traffic in Sec. 5.8.

\input{Chapter4/4_1_intro}

\input{Chapter4/4_5_related}

\input{Chapter4/4_2_background}

\input{Chapter4/4_3_data_collection}

\input{Chapter4/4_4_results}

\input{Chapter4/4_6_conclusion}

% !TEX root = ../toptesi-scudo-example.tex
% !TEX encoding = UTF-8 Unicode
%***********************************************************************
%***********************************Fifth Chapter
%***********************************************************************

\chapter{Electric Free Floating Car Sharing Mobility Simulator}
\label{chap:5_simulator}
	\graphicspath{{Chapter5/}}


%\input{Chapter5/5_0_abstract}

\section{Introduction}
In the previous chapters I widely described the dataset in I collected and characterized. However the collected data refers \emph{only} to internal combustion engine cars. Recalling that the main research question of my thesis is: \emph{It is possible design an electric Free Floating Car Sharing System?}, I need a tool able to replicate the FFCS customers' habits in a customizable model of electric FFCS system. 

In this chapter I describe the core-tool of my research: an event-driven trace-based electric FFCS simulator. In a nutshell, this software I developed is able to (i)  extract users pattern from data collected in chapter \ref{chap:2_dataset}; (ii) replicate the users patterns in a electrified scenario parametrized by the client; (iii) return several performance metrics describing the inputted electric FFCS setup.

In order to give a main idea, the simulator takes as input a real trace composed by a ordered set of rentals. More in details, the simulator creates the event-trace composed by \emph{Event} events. Each one of them carries temporal and spatial coordinates (when the event triggers) and it can represent a rental start or rental end. In this way, the simulator is able to mirrors the exact FFCS trips. 

Another input is the operative area composed by adjacent squares zones of 0.025 $km^2$. Then, each one of those zone can be equipped with a charging station, so the set of charging station and their placement is the third input. Complete the input set a car model and a fleet size.

By consuming the trace, the simulator computes trip by trip the amount of energy needed to travel the proper distance and moves the designed car from the starting point to the chosen destination. 

During the simulations, the software computes several metrics in order to measure to proper size the charging infrastructure and how it is reflected as user discomfort, i.e. in terms of number of plugging operation.

Those metrics are heavily influenced by the some environmental parameters like the number and the distribution of charging station. For this reason I proposed three placement strategies related to users' driving patterns. 

Moreover, an electric vehicle fleet needs a proper return policy to manage the battery state of charge. Indeed, the long charging time implies a smart car release, especially in zones having a charging station. The simulator takes in account this aspect too and compares different car return strategy.

This chapter is organized as follow: section \ref{sec:5_2_modelling} describes the the algorithm behind the simulator, section \ref{sec:5_3_mh_placement} illustrates the charging stations placement, section \ref{sec:5_4_return_policy} explains how I modelled the provider return policed that customers have to follow, section \ref{sec:5_5_kpi_scenario} explains the metrics taken in account and measured by the simulator and finally \ref{sec:5_6_conclusion} concludes the chapter proposing a work resume.

\section{Electric car sharing simulator}
\label{sec:5_2_modelling}

The goal is to study different design choices for electric car sharing systems. For this, I developed a flexible event-based simulator that allows us to compare different algorithms and tune their parameters while collecting metrics of interest. The simulator consumes a trace composed by a subset of rentals collected in \ref{chap:2_dataset}. In this way, by implementing an electric car consumption, I am able to model an electric FFCS provider that exactly replicate customers' temporal and spatial demand.
% Simulations are based on the actual traces collected from operative FFCS providers in each city. This allows to factor all spatial and temporal characteristics of actual FFCS customers habits.

\subsection{Simulation model}

The simulator replicates the behaviour of a fleet of electric cars, which are moving in the city. Each car is characterized by its location, and the current status of battery charge. The simulator takes as input a pre-recorded trace of rentals characterized by the start and end time, and initial and final geographic coordinates.

In more details, each trip $i \in \mathcal{I}$  is characterized by its start and end time, $t_{s}(i)$ and $t_{e}(i)$, and origin and destination coordinates, $o(i)$ and $d(i)$. For simplicity, I divide the city area into squared zones, of side 500\,m as before. Then, I associate with each position one and only one zone $O(i)=zone(o(i))$ and $D(i)=zone(d(i))$. We assume a charging station $cs$, composed of $k$ poles, can be placed at the center of a given zone $z\in \mathcal{Z}$, so either $cs(z)=1$ if the station is present, or $cs(z)=0$ otherwise. $N=\sum_{z\in \mathcal{Z}}cs(z)$ is the total number of zones equipped with charging stations, with 
$K=N\cdot k$ the total number of poles.


Additionally, it is present a set $\mathcal{A}$ of cars, with its cardinality $\left\vert{\mathcal{A}}\right\vert$ obtained by the trace. Each car $a\in \mathcal{A}$ at time $t$ is characterized by its position $p(a,t)$, its zone $P(a,t)=zone(p(a,t))$, and the residual battery capacity $c(a,t)\in[0,C]$, with $C$ being the maximum nominal capacity.

Generally speaking, the simulator processes each rental event $i$ in temporal order. When a \emph{rental-start} event $i$ is processed at time $t=t_{s}(i)$, the simulator chooses randomly one of the most charged available car in the closest zones to the initial position zone $O(i)$. In formulas, we get a car $\bar{a} \in \mathcal{A}$ such that:
\[
c(\bar{a},t) \geq c(\hat{a},t)\ \forall \hat{a} \in \argmin_{a \in A} {dist(O(i), P(a,t))}.
\]
Basically, the simulator mimics the normal behaviour of FFCS customers that use their smartphone to rent the closest car from their position and are worried about vehicle range~\cite{RangeAnxiety}. Notice that this behaviour is independent from whether the car is at a pole being charged or not.
Then, the simulator schedules the event \emph{rental-end} and it makes the car unrentable. When the rental ends fires, all the statistics about the rented car are updated (like battery consumption and new destination). Obviously, the simulator is able to manage all the events, like battery depletion or unavailable cars nearby the rental starts. 

In output, the simulator produces several statistics about system usage and user-related discomfort metrics related to the electric vehicle plugging procedures. 

\subsection{Modelling of rental event}
%A \emph{rental-end} event is then scheduled using the trace final time $t_{e}(i)$ and desired destination location $d(i)$.

When a \emph{rental-start} event $i$ is processed at time $t=t_{s}(i)$, and the simulator looks for a car in the initial position zone $O(i)$. If one or more cars are present, it selects (one among) the most charged car, i.e, get the car $a\in \mathcal{A}$ such that
\[
P(a,t) = O(i) \, \land \, c(a,t) \geq c(a',t)\ \forall a'\mid P(a',t) = O(i),
\]
independently whether the car is at a pole being charged or not.\footnote{We choose this policy because people are worried about vehicle range~\cite{RangeAnxiety}.}

If any car is available, the simulator selects the closest zone to $O(i)$ containing an available car, mimicking the normal behaviour of FFCS customers that use their smartphone to rent the closest car from their position. If any vehicle is present in the 8 eight neighbouring zones, the rental is marked as {\it infeasible}.
A \emph{rental-end} event is then scheduled using the trace final time $t_{e}(i)$ and location $d(i)$.

When car $a$ rental-end event is processed at time $t_{e}(i)$, the simulator makes as available the car in the real position $p(a,t_{e}(i))$. The arrival zones might correspond to the one present in the \emph{rental-end} event, or it might be necessary to manage a slightly user's re-routing due to vehicle plugging procedures. The policies to decide when and how plug the car are described in section \ref{sec:6_7_results}. Once the car is released, the simulator updates the battery State of Charge (SoC) by consuming an amount of energy proportional to the real trip distance:
\begin{eqnarray*}
	c(a,t_{e}(i)) = \nonumber \hspace{0cm}   \max{(c(a,t_{s}(i)) - Energy(p(a,t_{s}(i)), p(a,t_{e}(i))), 0)} 
\end{eqnarray*}

%\[
%c(a,t_{e}(i)) = \max{(c(a,t_{s}(i)) - Energy(p(a,t_{s}(i)), p(a,t_{e}(i))), 0)}
%\]
with $Energy(\cdot)$ that models the energy consumed to go from the car origin $p(a,t_{s}(i))$ to the car destination $p(a,t_{e}(i))$.
In case $c(a,t_{e}(i)) = 0$, the trip $i$ is declared {\it infeasible}. 
The discharged car $a$ still performs further trips, all marked as infeasible, until it reaches a charging station.

%\begin{algorithm}[H]
%	Events = List of  Events\\
%	Cars = List of car IDs\\
%	Zones = List of zones ID \\
%	Stations = Zones marked with a charging stations\\
%	Infeasible\_trips = 0 \\
%	Reroutings = 0\\
%	Walked\_distance = 0
%	Recharges = 0
%	\caption{Variables initialization}
%\end{algorithm}
%
%
%\begin{algorithm}[H]
%	
%	\For{Event in Events}
%	{
%		\If{Event = \textit{rental-start}}
%		{
%			current\_zone = Event.start\_zone\\
%			car\_id = get\_nearest\_car(current\_zone)\\
%			\If {car\_id == Null} {Infeasible\_trips+=1}
%			\Else{Cars[car\_id].set\_isrented()}
%		}
%		
%		\If{Event = \textit{rental-end}}
%		{
%			car\_id = Event.car\_id \\
%			current\_zone = Event.end\_zone\\
%			Cars[car\_id].update\_soc(Event.travelled\_distance) \\
%			\eIf{Cars[car\_id] <= 0} 
%			{
%				Infeasible\_trips+=1
%				Cars[car\_id].end\_zones = current\_zones
%			}
%			\eIf{Cars[car\_id] >= 0 & Cars[car\_id] <= $\pi$} 
%				{
%					new\_end\_zone = closest\_charging\_station(current\_zone)\\
%					Walked\_distasnce = distance(current\_zone, new\_end\_zone)\\
%					Reroute += 1
%					Rechages +=1
%					Cars[car\_id].end\_zone = new\_end\_zone 
%				}
%		}
%	
%		
%		
%	}
%	
%	
%	\caption{Simulation algorithm}
%\end{algorithm}

%\begin{algorithm}[H]
%	\For{Event in Events}
%	{
%		current\_plate = Booking.plate\\
%		current\_car = Cars[current\_plate]\\
%		Average\_SoC += current\_car.SoC\\
%		\If{current\_car was in charge}
%		{ 
%			zone = curret\_car.last\_booking.last\_final\_zone\\
%			Decrement the number of chargin car in Station[Zone]\\
%			current\_car.compute\_recharged\_power\\
%			charges += 1
%		}
%		arrival\_zone = Booking.final\_zone\\
%		current\_car.compute\_consumption(Booking.distance)\\
%		
%		\eIf{current\_car.current\_capacity > 0}
%		{
%			\eIf{arrival\_zon is a Charig Station and \\
%				Station[arrival\_zone].availabe\_spots > 0}
%			{
%				Station[arrival\_zone].availabe\_spots -= 1\\
%				current\_car.in\_charge = True\\
%				current\_car.save\_booking(Booking)
%			}
%			{
%				current\_car.in\_charge = False\\
%				current\_car.save\_booking(Booking)
%			}
%		}
%		{
%			Discharge +=1
%		}
%	}%% end for
%	Average\_SoC = Average\_SoC / len(Bookings) \\
%	Charges = Charges / len(Bookings) \\
%	Discharge = Discharge / len(Bookings)\\
%	
%	\caption{Simulation algorithm}


\section{Meta-Heuristic Charging Stations Placement}
\label{sec:5_3_mh_placement}
In this sections I explain the charging station placement algorithm. The output of this algorithm is one of the most relevant environmental variables that will be deeply studied and analysed in chapters \ref{chap:6_4cities} and \ref{chap:7_cs_optimization}. The main idea behind this algorithm to rank each zone (defined in section \ref{sec:5_2_modelling}) according a the users' travelling patterns and, then equip zones having the highest values.


\subsection{Problem formalization}
Given a number of charging station $N$, the first objective is to place them in the city area so to let all rentals feasible, i.e., to find a charging stations placement so that
\[
c(a,t_e(i))>0\ \forall a \in \mathcal{A}, \forall  i \in \mathcal{I}
\]
Since I do not make any assumption on the set of trips $\mathcal{I}$, I cannot know a-priori if a solution exists and provide an analytical general solution. Moreover the number of candidate solutions increases as the binomial coefficient ${\left\vert{\mathcal{Z}}\right\vert}\choose\ N$, making ineffective to numerically compute all possibilities. Instead, I will provide a class of greedy algorithms and analyse the performance in our specific cases of $\mathcal{I}$.
In details, each zone $z\in\mathcal{Z}$ is assigned a likelihood $l_z \geq 0$.
We then solve the problem of finding the subset of $N$ zones that maximizes the total likelihood. In formulas, 
$$\max \sum_{z\in\mathcal{Z}} cs(z)l_z$$

subject to:
$$\sum_{z\in\mathcal{Z}} cs(z) = N$$
$$cs(z)\in \{0,1\},  \forall z \in \mathcal{Z}$$

The above optimization problem can be solved by greedily choosing the top $N$ zones, ordered in decreasing likelihood. We compare the performance of different placement algorithms based on different definition of the likelihood.
\begin{itemize}
	\item{\it Random placement}: $l_z$ is an independent and identical distributed random uniform variable, so that charging stations result placed at random;
	\item{\it Average parking time}: $l_z$ is the average parking duration in $z$ as recorded in the trace;
	\item{\it Total number of parkings}: $l_z$ is the total number of parking events recorded in $z$ in the trace;
	\item{\it Total parking time}: $l_z$ is the total parking time accumulated in $z$ by all cars recorded in the trace. In each zone, it is the product of the two previous metrics.
\end{itemize}
Those heuristics are driven by the intuition that placing charging stations in those zones where cars are parked for long time (average parking time) or frequently parked (total number of parkings) could improve system performance.


\begin{figure}[th]
	\centering     %%% not \center
	\subfloat[\centering Turin]{{\includegraphics[width=0.24\columnwidth]{images_pdf/Torino_AvgTime.pdf}}\label{fig:5_4_ap_turin}}
	\quad
	\subfloat[\centering Vancouver]{{\includegraphics[width=0.30\columnwidth]{images_pdf/Vancouver_AvgTime.pdf}}\label{fig:5_4_ap_vancouver}}
	\quad
	\subfloat[\centering Berlin]{{\includegraphics[width=0.39\columnwidth]{images_pdf/Berlino_AvgTime.pdf}}\label{fig:5_4_ap_berlin}}%
	\caption{Distribution of average parking time in Turin, Vancouver and Berlin}
	\label{fig:5_4_heatmap_avgparking}
\end{figure}

I order to show the differences between the likelihoods $l_z$ criteria, figures \ref{fig:5_4_heatmap_avgparking} where $l_z$ is depicted for Turin (\ref{fig:5_4_ap_turin}), Vancouver (\ref{fig:5_4_ap_vancouver}) and Berlin (\ref{fig:5_4_ap_berlin}). The first two cities were deeply characterized in chapters \ref{chap:3_charact} and \ref{chap:4_cs_comparison}, while Berlin, as I will show, presents some interesting spatial distribution. In all the figures, in particular, the more the zone is red, the higher is $l_z$. It means that the \emph{redest} zones will be the first to host a charging station.

In first approach it is possible to se how, in all figures, the heuristic \textit{Average parking time} is mainly spread in city peripheries. It means that the cars spend a lot of times parked far from city centre. This peculiarity can be imputed to commuting patterns: as figure \ref{fig:3_4_bookingsweek} points out, two peaks are present in the users' demand. In particular the evening peak catches the back-home commuting which, usually is directed to high-density residential area located in periphery. This, joint with the low business-days night demand, leads to users to leave cars parked in that areas all night long.


\begin{figure}[th]
	\centering     %%% not \center
	\subfloat[\centering Turin]{{\includegraphics[width=0.24\columnwidth]{images_pdf/Torino_NParkings.pdf}}\label{fig:5_4_np_turin}}
	\quad
	\subfloat[\centering Vancouver]{{\includegraphics[width=0.30\columnwidth]{images_pdf/Vancouver_NParkings.pdf}}\label{fig:5_4_np_vancouver}}
	\quad
	\subfloat[\centering Berlin]{{\includegraphics[width=0.40\columnwidth]{images_pdf/Berlino_NParkings.pdf}}\label{fig:5_4_np_berlin}}%
	\caption{Distribution of number of parking in Turin, Vancouver and Berlin}
	\label{fig:5_4_heatmap_numparking}
\end{figure}

Figure \ref{fig:5_4_heatmap_numparking} depicts the number of parking in each zone, for Turin (\ref{fig:5_4_np_turin}), Vancouver (\ref{fig:5_4_np_vancouver}) and Berlin (\ref{fig:5_4_np_berlin}). Reminding that more parkings means an higher zone attractiveness,
 it is possible to notice how the zones with the highest number of parking concentration zones are delimited in particular areas. For example figure \ref{fig:5_4_np_turin} shows how most frequented areas are downtown in correspondence of the two main train stations and the airport. A similar pattern can be spot in figure \ref{fig:5_4_np_vancouver}. Contrary, Berlin presents at least three attractive areas. This is mainly due to the biggest operative area and, probably, to the differentiation of business areas.
 
For completeness, I report in figure \ref{fig:5_4_heatmap_sumtime} the \textit{Total parking time} likelihood. It appears to smooth the behaviour of the previous two metrics.

\begin{figure}[th]
	\centering     %%% not \center
	\subfloat[\centering Turin]{{\includegraphics[width=0.24\columnwidth]{images_pdf/Torino_SumTime.pdf}}\label{fig:5_4_st_turin}}
	\quad
	\subfloat[\centering Vancouver]{{\includegraphics[width=0.30\columnwidth]{images_pdf/Vancouver_SumTime.pdf}}\label{fig:5_4_st_vancouver}}
	\quad
	\subfloat[\centering Berlin]{{\includegraphics[width=0.39\columnwidth]{images_pdf/Berlino_SumTime.pdf}}\label{fig:5_4_st_berlin}}%
	\caption{Distribution of total parking time in Turin, Vancouver and Berlin}
	\label{fig:5_4_heatmap_sumtime}
\end{figure} 

This brief catheterization shows how  different cities can have different spatial characterization and thus different charging station placements. However those characterization will be deepened in chapter \ref{chap:6_4cities}.


\section{Car return policies}
\label{sec:5_4_return_policy}
One of the most challenging points of electric FFCS is to deal with the discomfort derived by plug in operations. This operation is more time consuming with compared to the normal filling up procedure of combustion engine cars. Therefore, the providers have to deal with users' selfishness and trying to stimulate their willingness.  
%In this Section, we define different policies that the electric FFCS may enforce, and different probabilistic behaviors of customers. 

%\subsection{Car charging policies}
When returning the car, the customer may connect the car to a pole in a station, hence charging the car battery and possibly deviating the real destination from the desired one. I modelled the following policies:
\begin{itemize}
	\item{\it Free Floating}: the customer must connect the car to a charging pole if and only if it is available in the desired final zone $D(i)$;
	\item{\it Forced}: cars must be connected to a pole when the percentage of battery charge at the end of the rental $i$ would go below a certain threshold $\pi$, i.e., $(c(a,t_{s}(i)) - Energy(p(a,t_{s}(i)), d(i))) \cdot 100/ C\leq  \pi $. This implies the customer can be \textit{rerouted} to the closest zone to the desired one $d(i)$, if no free pole exists in the zone; %Battery consumption takes into account the additional traveled distance.
	\item{\it Hybrid}: the customers follow the forced policy; they may  also choose to connect to a charging pole available in the desired ending zone $D(i)$ with probability $w\in [0,1]$;
	%if the $c(a,t)\leq\pi$, cars must be returned to the closest recharging zone to $d(i)$.
\end{itemize}

The \textit{Free Floating} policy never obliges the customer to bring the car far from the desired ending location, even in case battery is close to exhaustion. It used as benchmark, in order to understand until when the users might rent car without any restriction/
\textit{Forced} mandates to connect cars to a charge station only when energy runs low, thus trying to protect from battery exhaustion.
\textit{Hybrid} introduces the level of customers willingness to collaborate, named with $w$. $w=0$ is equivalent to the Forced policy, while $w=1$ adds to the Forced policy the Free Floating policy,  thus always connecting the car to a charging pole if available in their final position zone. The users' willingness should be $w$ should be intended as the probability that a user can collaborate with the provider, dropping the car in a charging station. The $w$ variability can by justified like provider incentive bonus like car2go free minutes after a car filling up.



\section{Key Performance Indicators and Simulation Scenario}
\label{sec:5_5_kpi_scenario}
In this section, I describe which are the simulation outputs and the scenario with which I performed the analyses. In particular I focused the attention on minimum requirements to system sustainability and measuring of users discomfort. 

\subsection{Performance metrics and parameters}

The simulator measures metrics that are  key to assess in the quality of experience for the customers:
\begin{itemize}
	\item \emph{Infeasible trip}: measures if a trip $i$ performed by a car $a$ ends with a completely discharged battery, i.e., when $c(a,t_{e}(i))= 0$;
	\item \emph{Charge event}: indicates a trip $i$ that ends with putting in charge the car, implying the burden to drive to the pole position, and plug the car;
	\item \emph{Reroute event}: a trip $i$ where the customer is rerouted to a zone different from the  desired destination because forced to charge the car $a$, i.e., $P(a,t_{e}(i))\neq D(i)$;
	\item \emph{Walk distance}: distance between the desired final location $d(i)$ and the actual final position $p(a,t_{end}(i))$.
\end{itemize}

The number of infeasible trips are critical, and the system shall be engineered so that they never happen. Other performance metrics shall be minimized. 
In addition to the above metrics, the simulator collects statistics about car battery charge level $c(a,t)$, and fraction of time a battery stays under charge.

\subsection{Simulation scenario}

I use this simulator to study the impact on the number of zones that are equipped with charging stations $N$, and the number of poles $k$ of each charging station.

I consider in each city a fleet that has a number of cars equal to the one observed in the trace. Electric cars have the same nominal characteristics as the Smart ForTwo Electric Drive, i.e., $17.6\,kWh$ battery, for $135\,km$ of range, with a discharge curve $Energy()$ that is proportional to the travelled distance ($12.9\,kWh/100\,km$). \footnote{\url{https://www.smart.com/uk/en/index/smart-electric-drive.html}} 
Charging stations have $k=4$ low power ($2\,kW$) poles each. These are cheap to install and a good compromise between costs, power requested, and occupied road section. We model a simple linear charge profile (complete charge in 8 hours and 50 minutes in our case). At last, the initial car position, only affecting the simulation transient, is chosen randomly.

The simulator, written in Python, takes less then 5 seconds to complete a single simulation for a given city and parameter set. 
Due to the large number of simulations, we run them in parallel. Each simulation produces 100\,MB of detailed logs, that we process on a Big Data cluster of 30 nodes using PySpark.%\footnote{\url{http://spark.apache.org/docs/latest/api/python/\#}}.

\section{Conclusions}\label{key}
\label{sec:5_6_conclusion}
In this chapter I described a FFCS electric mobility simulator I developed. Starting from the data collected with the software described in chapter \ref{chap:2_dataset} I created trace of rental events, describing the system allocated users' demand. 

More in details, the simulator allocates a set of cars, characterized by battery capacity and power consumption per kilometres. Then consumes the rental trace, marking the car unavailable after a \textit{rental-start} event and updating the final battery state of charge when a \textit{rental-end} event is processed. Moreover, the simulator is in charge to place the charging station according three heuristics: random, preferring zones having a grater parking time and zones having the higher number of parkings. Finally it takes in account the different policies with which the users have to return the car.

When the trace is consumed, this simulator computes several key performance indicators measuring the proper system infrastructure allocation and users' discomfort to deal with an electric vehcile
% !TEX root = ../toptesi-scudo-example.tex
% !TEX encoding = UTF-8 Unicode
%***********************************************************************
%***********************************fifth Chapter
%***********************************************************************

\chapter{A Data Driven Approach for Electric FFCS System Design}
\label{chap:6_4cities}
	\graphicspath{{Chapter6/}}

%\section{Abstract}
%
%Car sharing is a popular means of transport in smart cities, with the free floating paradigm which lets the customers autonomously pick and drop available cars. This freedom impose several challenges and trade-off in the design of an electrical version of this type of car sharing. 
%
%In this paper I study the design of an electric free floating car sharing system for four cities by leveraging actual rental traces. I analyze via accurate simulations through the simulator described in chapter \ref{chap:5_simulator} the impact of i) the charging station placement, and ii) return policies.
%The traces composed by millions of rentals in different cities world wide located, captures the non-stationary and highly dynamic nature of usage patterns of customers.
%
%Considering charging station placement, it is possible to demonstrate how it is better to place charging stations within popular parking area (e.g., downtown), even if parking duration is short. I show how by smart placing and handful number of charging stations (below 10\% of city zones) the system can self-sustain itself. 
%These results are obtained also thanks to customers' willingness to collaborate by voluntarily returning the car to a nearby charging station.
%
%The results help car sharing provider comparing possible alternative design solution, i.e.,  the adoption of simple relocation policies that would move cars that need a charge only, a promising solution to limit discomfort for customers due to rerouting. The same could be achieved by considering giving incentives to customers. 

This chapter refers mostly two works: \textit{"Free floating electric car sharing in smart cities: Data driven system dimensioning"} published in the IEEE International Conference on Smart Computing (SMARTCOMP) in July 2018 (\cite{taormina}) and \textit{"Free Floating Electric Car Sharing: A Data Driven Approach for System Design"} published in the IEEE Transactions on Intelligent Transportation Systems journal in August 2019 (\cite{coccacar})


\section{Introduction}
\label{sec:6_1_intro}

Nowadays mobility is a very important challenge for our society, with strong implications on pollution in large cities where more eco-sustainable solutions are positively seen as a mean to improve the current situation.
Along with the usage of public transport, the sharing mobility such as bike sharing, car pooling and car sharing, can help to address this problem. In this chapter,  I focus on the design of an electric car sharing system, where customers rent a car for moving within the city limits for short periods of time. I focus on the so called Free Floating Car Sharing (FFCS) system where customers are free to pick and return the car wherever they like, inside a geo-fenced area.
Electric car sharing systems need an infrastructure of recharging stations, whose design requires ingenuity~\cite{PlacementAndPowergrid,placementAustin,mipCSPpechino}. 


Data is fundamental to answer these design questions. In this work, I base the study on the availability of millions of actual rentals I collected from currently in use FFCS systems as reported in chapter \ref{chap:2_dataset}. Im this chapter I consider Turin and Vancouver (charchetrized in chapters \ref{chap:3_charact} and \ref{chap:4_cs_comparison}) plus Berlin(Germany) and Milan(Italy) which in the dataset are the cities having the biggest number of rentals.
The data naturally factors the non-stationarity of FFCS systems including millions of actual rentals. Armed with this, I study and compare the performance of a hypothetical equivalent car sharing system based on electric vehicles.  While in the past some works have proposed solutions for the design of electric FFCS~\cite{FM15,WB15} and for a smart placement of charging stations, this work is among the first to take a data driven approach for the design of electric car FFCS systems~\cite{ChargingStationForVehicularNetworks,mipCSPpechino,PlacementAndPowergrid,placementAustin}.

First, I characterize how customers actually use FFCS transport means in different cities and countries. Results show a similar usage with an high utilization during commuting time and very different spatial distributions. Rental duration and driving distance are quite short (less than 20-30 minutes, for less than 5\,km in median). More interestingly, it is possible to observe peripheral zones where cars are left parked for long time, and busy areas where instead the parking duration is much short and dynamic.

Armed with these facts, I compare charging station placement policies introduced in section \ref{sec:5_3_mh_placement} that exploit the knowledge of typical parking zones and duration. I first assume a pure Free Floating system, where customers return the car in a charging station only if present at their actual destination.
Results show that placing the charging stations in those areas where cars stay parked for long time performs badly. Instead, placing charging station in those areas where cars are frequently parked and rented, e.g., near train stations and working areas, guarantees much better performance. This is consistent in all cities.

Next, I study different return policies introduced in section \ref{sec:5_4_return_policy}, where customers are asked to return the car to a charging station in case the battery level decreases below a minimum threshold. This collaborative policy reduces the cost of the charging infrastructure by a factor of 2 or more with respect to a pure opportunistic free floating solutions. Equipping just 8\% of charging zones with 4 poles of 2\,kW would guarantee a electric car FFCS equivalent to the one currently in use.
This with minimal impact on the customer satisfaction, measured by the number of times customers are forced to drive to a charging station and the distance they have to walk back to the desired destination.

At last, I compare system design alternatives to check whether is better to place a lot of charging poles in very few areas, or rather to spread a lot charging stations with few poles in many areas. Results demonstrate that both extreme solutions perform badly, with best performance when installing charging stations with  5 to 20 poles in popular areas. This has benefits also on the power grid used to supply power to the recharging areas.


After quickly discussing related work in section~\ref{sec:6_2_related}, I present and characterize data in section~\ref{sec:6_3_data}, 
%and the simulation model and tool in section~\ref{sec:6_4_Modelling}. 
section~\ref{sec:6_6_freefloating} discusses the impact of charging stations placement policies, while section~\ref{sec:6_7_results} compares return policies. Section~\ref{sec:6_8_resPoles} presents the impact of concentrating or spreading charging stations in the city. Finally section~\ref{sec:6_9_conclusion} concludes the chapter.


\input{Chapter6/6_2_related_work}

\input{Chapter6/6_3_data}

%\input{Chapter6/6_5_modelling}

\input{Chapter6/6_6_placement_results}

\input{Chapter6/6_7_return_results}

\input{Chapter6/6_8_poles_results}

\section{Conclusions}
\label{sec:6_9_conclusion}

Designing an electric vehicle free floating car sharing system leads to many interesting problems and trade-offs.
In this chapter, I built on actual rental traces to study via accurate simulations the impact of i) the charging station placement, and ii) return policies. 
Considering charging station placement, I demonstrated that it is better to place charging stations within popular parking area (e.g., downtown), even if parking duration is short.

The analyses have shown that a FFCS solution with electric vehicles can almost sustain itself, even with very few charging stations (8-10\% of city zones). These results are obtained also thanks to customers' collaboration by returning the car to a nearby charging station, or whenever the battery level drops below a target threshold.

Car sharing providers shall take into account the trade-off between usability, costs and benefits for the customers. 
The results hint for possible alternative design solution, i.e.,  the adoption of simple relocation policies that would move cars that need a charge only, a promising solution to limit discomfort for customers due to re-routing enforcement. This could be achieved by considering giving incentives to customers.


% !TEX root = ../thesis.tex
% !TEX encoding = UTF-8 Unicode
%***********************************************************************
%***********************************sixth Chapter
%***********************************************************************

\chapter{Data-Driven Optimization of Charging Station Placement in Electric FFCS}
\label{chap:7_cs_optimization}
	\graphicspath{{Chapter7/}}

%\section{Abstract}
%%Self-contained abstract of no more than 100 words MA HO VISTO CHE NEGLI ALTRI PAPER NE USANO CIRCA 180, outlining the aims, scope and conclusion of the paper. Three to five keywords must be included.
%%Ora siamo a 209
%
%In this work I consider the design of a Free Floating Car Sharing (FFCS) system based on Electric Vehicles. I face the problems of finding the optimal placement of charging stations, and the design of smart car return policies, i.e, how many and where to place charging stations, and whether to ask or not customers to return the car to a charging pole.
%
%I leverage actual data containing rentals performed by Car2Go customers described in chapter \ref{chap:2_dataset}. I obtain information for several months worth of actual trips in the city of Turin, this use case.
%Via trace driven simulations, I replay the exact same trips while simulating electric car based FFCS, to accurately gauge battery discharging and recharging.
%With this, I compare different charging station placements, also driven by optimisation algorithms. Moreover, I observe the impact of collaborative or selfish car return policies.
%
%Results are surprisingly: just as few as 13 charging stations (52 poles) guarantee a fleet of 377 vehicles running in a 1 million inhabitant city to work flawlessly, with limited customer's discomfort.
%I believe this data driven methodology may help researchers and car sharing providers discerning different design solutions in order to find a lower-bound reference to setup an charging infrastructure for a FFCS system.

This chapter is mostly taken from \textit{"Data driven optimization of charging station placement for EV free floating car sharing"} published in the $21^{st}$ International Conference on Intelligent Transportation Systems in November 2018 (\cite{maui}) and \textit{"Free floating electric car sharing design: Data driven optimisation"} published on Pervasive and Mobile Computing journal in April 2019 (\cite{coccaopt})
  

\input{Chapter7/7_1_intro}

\input{Chapter7/7_2_related_work}

\input{Chapter7/7_3_data}

\input{Chapter7/7_4_modelling}

\input{Chapter7/7_5_strategies_cs}

%\input{Chapter7/7_6_heuristic_placement_results}

\input{Chapter7/7_7a_heuristic_vs_optimized}

\input{Chapter7/7_7b_needed}

\input{Chapter7/7_8_conclusion}
% !TEX root = ../toptesi-scudo-example.tex
% !TEX encoding = UTF-8 Unicode
%***********************************************************************
%***********************************Seventh Chapter
%***********************************************************************

\chapter{FFCS Usage Prediction with Open Socio-Demographic Data}
\label{chap:8_prediction}
	\graphicspath{{Chapter8/}}



%\section{Abstract}
%Free Floating Car Sharing (FFCS) services are a flexible alternative to car ownership. These transportation services show highly dynamic usage both over different hours of the day, and across different city areas. 
%In this work, we study the problem of predicting FFCS demand patterns -- a problem of great importance to an adequate provisioning of the service. We tackle both the prediction of the demand i) over time and ii) over space. 
%We rely on months of real FFCS rides in Vancouver, which constitute our ground truth. We enrich this data with detailed socio-demographic information obtained from large open-data repositories to predict usage patterns. 
%Our aim is to offer a thorough comparison of several machine learning algorithms in terms of accuracy and easiness of training, and to assess the effectiveness of current state-of-art approaches to address the prediction problem.
%Our results show that it is possible to predict the future usage with relative errors down to 10\%, and the spatial prediction can be estimated with relative errors of about 40\%.
%Our study also uncovered the socio-demographic features that most strongly correlate with FFCS usage, providing interesting insights for providers interested into opening service in new regions.

This work is mainly extracted from my paper \textit{On Car-Sharing Usage Prediction with Open Socio-Demographic Data}, published on the journal Electronics on January 2020 (\cite{cocca2020predictions}). The entire work was carried in collaboration with the Universidade Federal do Minas Gerais, Belo Horizonte, Brazil. My contribution are mainly related data collection, data augmenting and spatial analyses in sections \ref{sec:8_3_data_collection}, \ref{sec:8_3_datasetoverview} and \ref{sec:8_5_spatial analyses}.






\input{Chapter8/8_1_introduction}

\input{Chapter8/8_2_related_works}

\input{Chapter8/8_3_data_collecting_and_carachterization}

\input{Chapter8/8_4_temporal_analyses}

\input{Chapter8/8_5_spatyal_analyses}

\input{Chapter8/8_6_conclusion}
%\include{Chapter9/9_relocation}
% !TEX root = ../toptesi-scudo-example.tex
% !TEX encoding = UTF-8 Unicode
%***********************************************************************
%***********************************tenth Chapter
%***********************************************************************

\chapter{Scalability of Electric FFCS in Smart Cities}
\label{chap:10_scalability}
	\graphicspath{{Chapter10/}}


%\section{abstract}
%\label{sec:10_0_abstract}
%
%In this chapter I analyze which are the design options that would impact a free floating electric car sharing system performance and costs, studying how the system would scale with an increase in the intensity of the demand.
%I consider the case study of the city of Turin, for which I leverage hundred of thousands of actual rentals from a (combustion-based) car sharing system to derive an accurate demand model. Armed with this, I consider the transition to  electric cars and the need to deploy a charging station infrastructure.
%
%Using a realistic simulator, I present the impact of system design options, like the number of charging poles, their allotment, and the number of cars. I first consider performance indicators, like fraction of satisfied demand and working hours system has to spend to bring to charge vehicles. Then I map these figures into revenues and costs, projecting economical indicators. 
%At last, I investigate the scalability of the whole system, i.e., how performance and costs scale when the demand increases.
%The results show that concentrating the charging stations in key places is instrumental to optimize car distribution in the city to better intercept the demand.
%Considering system scalability, the charging infrastructure must intuitively grow proportionally with the mobility demand.
%Interestingly instead, the fleet size can grow much slower, showing some nice economy of scale gains.

This chapter is mostly taken from \textit{"On Scalability of Electric Car Sharing in Smart Cities"} (\cite{barulli2020scalability}) published in the 2020 IEEE International Smart Cities Conference (ISC2) in September 2020.


\input{Chapter10/10_1-introduction}

\input{Chapter10/10_2-related}

\input{Chapter10/10_3-dataset}

\input{Chapter10/10_4-modeling}

\input{Chapter10/10_5-results}

\input{Chapter10/10_6-conclusions}
% !TEX root = ../toptesi-scudo-example.tex
% !TEX encoding = UTF-8 Unicode
%***********************************************************************
%***********************************fifth Chapter
%***********************************************************************

\chapter{Conclusions and Future Works}
\label{chap:11_conclusion}
	\graphicspath{{Chapter11/}}



In this thesis, I proposed a data-driven pipeline to study the conversion from internal combustion engine electric vehicles in FFCS. Strongly relying on real FFCS data, optimization algorithms, and machine learning techniques, I depicted a pipeline able to evaluate the performances of electric FFCS in hot and cold start-up scenarios. 

In the first part of my thesis, I described the data gathering, scraping real users' ride from operative combustion engine FFCS. Then I characterized the users' habits in 2017 for two FFCS providers: car2go and Enjoy. I highlighted the difference between one-way CS, two-ways CS, and FFCS, which pointed out how the users prefer flexibility over costs. Once all the data are consolidated I developed an electric FFCS mobility simulator. It is an event-based trace-driven simulator. It takes as input the list of rentals events, the infrastructure setups (in terms of charging station placements, car consumptions), and the provider's fleet management policies. It replicates users' patterns and returns as output the metrics describe system performance (complete battery depletion, cars unavailability) and users' discomfort caused car plugging procedures. Armed with this, I initially investigated the best electric FFCS setup varying case of study city (and thus users' patterns), finding how an electric FFCS is sustainable with few charging stations. Next, I improved the solution of the previous step by running several optimization algorithms maximizing both system resilience and minimizing the users' discomfort. Then, I moved my attention to users' ride predictability computing how weather and socio-economics aspects are related to the users' trips. Finally, I simulated possible electric FFCS growth targeting which are the most sensible parameters and how profitable an electric FFCS may be.

The results are surprising. In particular, the data-driven charging station placement showed how the current demand might be sustained with only 104 charging poles in a city of about 1 million inhabitants like Turin. This estimate may be still reduced with ad-hoc algorithms at 72 poles. Finally, the scalability studies (made open source) may lead FFCS providers and policy makers aware of all the benefits of electric mobility.

However, the research presents several limitations. In particular, the dataset is composed of passive measurements, the reason for that it is impossible to distinguish users' trips and unallocated missed trips. The software does not really know when the users did not find an available vehicle nearby. Moreover, the real travelled distances are unknown, due to the anonymization related to privacy. This should not really impact macro analyses but it becomes sensible when it necessary to perform neighbourhood-grained optimizations. This dissertation does not take into account fleet management (relocation). This topic opens several research threads, that need a pragmatic formulation and sophisticated upgrade to the software. Finally, the recent COVID-19 pandemic changed drastically both private and shared users' patterns opening new questions on mobility sustainment in smart cities.

In general, I believe my thesis proposes a pragmatic methodology to study the electric (r)evolution of shared mobility. It will be very challenging and stimulating for me to face the unsolved problem in my future research career.



%% Numbered appendices remain in the main matter...
%\appendix
%% !TEX root = ../thesis.tex
% !TEX encoding = UTF-8 Unicode
% ******************************* Thesis Appendix A 

\chapter{List Of Publications}
\section{Journal Publications}
\begin{itemize}
	\item \textbf{Cocca, M.}, Giordano, D., Mellia, M., & Vassio, L. (2019). Free floating electric car sharing design: Data driven optimisation. Pervasive and Mobile Computing, 55, 59-75.
	
	\item Alencar, V. A., Rooke, F., \textbf{Cocca, M.}, Vassio, L., Almeida, J., & Vieira, A. B. (2019). Characterizing client usage patterns and service demand for car-sharing systems. Information Systems, 101448.
	
	\item \textbf{Cocca, M.}, Giordano, D., Mellia, M., & Vassio, L. (2019). Free Floating Electric Car Sharing: A Data Driven Approach for System Design. IEEE Transactions on Intelligent Transportation Systems, 20(12), 4691-4703.
	
	\item \textbf{Cocca, M.}, Teixeira, D., Vassio, L., Mellia, M., Almeida, J. M., & Couto da Silva, A. P. (2020). On Car-Sharing Usage Prediction with Open Socio-Demographic Data. Electronics, 9(1), 72.
\end{itemize}



\section{Conference Publications}
\begin{itemize}
	\item Ciociola, A., \textbf{Cocca, M.}, Giordano, D., Mellia, M., Morichetta, A., Putina, A., & Salutari, F. (2017, August). UMAP: Urban mobility analysis platform to harvest car sharing data. In 2017 IEEE SmartWorld, Ubiquitous Intelligence & Computing, Advanced & Trusted Computed, Scalable Computing & Communications, Cloud & Big Data Computing, Internet of People and Smart City Innovation (SmartWorld/SCALCOM/UIC/ATC/CBDCom/IOP/SCI) (pp. 1-8). IEEE.
	
	\item \textbf{Cocca, M.}, Giordano, D., Mellia, M., & Vassio, L. (2018, November). Data driven optimization of charging station placement for EV free floating car sharing. In 2018 21st International Conference on Intelligent Transportation Systems (ITSC) (pp. 2490-2495). IEEE.
	
	\item \textbf{Cocca, M.}, Giordano, D., Mellia, M., & Vassio, L. (2018, June). Free floating electric car sharing in smart cities: Data driven system dimensioning. In 2018 IEEE International Conference on Smart Computing (SMARTCOMP) (pp. 171-178). IEEE.
	
	\item Ciociola, A.,\textbf{Cocca, M.}, Giordano, D., Vassio, L., & Mellia, M. (2020, September). E-Scooter Sharing: Leveraging Open Data for System Design. In 2020 IEEE/ACM 24th International Symposium on Distributed Simulation and Real Time Applications (DS-RT) (pp. 1-8). IEEE.
	
	\item Barulli, M., Ciociola, A., \textbf{Cocca, M.}, Vassio, L., Giordano, D., & Mellia, M. On Scalability of Electric Car Sharing in Smart Cities. In 2020 IEEE International Smart Cities Conference (ISC2) (pp. 1-8). IEEE.
	
\end{itemize}








%\include{Appendix2/appendix2}

\backmatter% here begins the back matter
% ... otherwise a single appendix may stay here
%\include{References/biblio}

% Do not use this command if you did not prepare a nomenclature
% database by means of the suitable \nomenclature command and its
% arguments, as we did in chapter 2 of this example thesis.
\printnomencl

% In this example we use \nocite{*} in order to typeset the whole
% contents of the bibliographic database. Normally this is not
% necessary and it's better to let biber extract from the database
% only the cited works
%\nocite{*}


\printbibliography[heading=bibintoc]

% Do not use this command if you did not set the \makeindex switch
% in the preamble.

\printindex

\end{document}

