%\section{Abstract}
%\label{sec:4_0_abstract}

%The understanding of the mobility on urban spaces is useful for the creation of smarter and sustainable cities. However, getting data about urban mobility is challenging, since only a few companies have access to accurate and updated data, that is also privacy-sensitive. 
%
%In this work, we characterize three distinct car-sharing systems which operate in Vancouver (Canada) and nearby regions, gathering data for more than one year. The study uncovers patterns of users' habits and demands for these services. 
%We highlight the common characteristics and the main differences among car-sharing systems. Finally, we believe our study and data is useful for generating realistic synthetic workloads.

This chapter refers mostly the paper \textit{Characterizing client usage patterns and service demand for car-sharing systems} \cite{VancouverCS}, published on the Journal Information Systems, available online since October 11, 2019. My contribution is mainly focused in all the aspects about car2go analyses. 

%Danilo's declaration
%The work I present in this chapter has been developed with the DataBase an Data Mining Group (DBDMG) at the Politecnico di Torino. I joined this group as a network expert to helped them on the validation of the SeLINA tool. In particular, I offered my skill to verify if SeLINA is able to highlight network traffic anomalies. My contributions will be presented in Sec. 5.7 related to the study of the YouTube anomaly previously highlighted in Sec. 2.6.1, and in the analysis of peer-to-peer traffic in Sec. 5.8.